\chapter{Conclusions} \label{ch:conclusion}

The complexity of the modern era supply chain makes it vulnerable to all kinds of disruption. The network structure of supply chain becomes disadvantageous during a disruption as the effects of the disruption can spread through the phenomenon of ripple effect. The disruptions can come from a various number of sources. The comprehension of these sources can help in finding out what capabilities of the supply chain it will most likely target. Once the most vulnerable capabilities are identified, it can help in devising the mitigation strategies. If there is no information about the vulnerable capabilities, a lot of time is wasted in the assessment of the damage and searching for the area for improvement that would aid in recovery of the disrupted facility of the supply chain. The contingency planning allocates a different facility from the available list based on the parameters. These parameters can be altered as per different types of supply chains. The base model created is a very generic model which can be modified as per the need of the company. The number of agents depends on the number of facilities present in the supply chain. This model can be applied to real and complex supply chains and the best results can be obtained in just a short span of time. 

It is important to design the supply chain keeping in mind these disruptions. The inclusion of the resilience and robustness components as mentioned in this research will make the supply chain able to recover from the damages of these disruptions in as less duration of time as possible. The results obtained from the methodology justifies the importance of these components from the supply chain design perspective. The resilience percentage of the supply chain is increased by minimum 10 \%. The time lost in the identification of the capabilities during a disruption is brought down by a considerable amount. This in turn reduces the time difference between the time to initiate the recovery process and and the time to recovery. The resilience triangle mentioned earlier in the study, is shortened by this process. The future scope of the study would be to implement the logistics capability into the model. The study quantifies the need for implementation of resiliency and robustness into the supply chain for having a competitive advantage in the market and also makes the supply chain more responsive. 




