\chapter{Literature Review} \label{ch:litreview}

There has been quite a lot of research in the comprehension of the concepts of resiliency and robustness in the supply chain and its significance. Many firms have faced the adverse effects of disruption on the performance of their supply chains. The loss in terms of money, time, and the trust of the customers have forced the managers and owners to turn to the recourse of having resiliency and robustness incorporated in their initial design phase of the supply chain. 

The management of disruptions and the decisions that are made during the occurrence of a disruption plays a pivotal role in defining the performance and quality of the system. This can prove to have an advantage over the competitors. An example of this is the approach and decision-making skills of Nokia when the radio frequency chip manufacturing plant of Philips electronics was caught on fire. Even though the fire was contained and taken care of in less than 10 minutes, the effects of the fire had a heavy impact on the chips that were present in the storage area. Due to the water and smoke, the chips in the storage were damaged and rendered useless. Philips was a major chip suppler to Nokia and their competitors Ericsson. The Philips management declared to their customers (Nokia and Ericsson) that the disruption would be taken care of within the span of a week. While the management at Ericsson were unaware of the impact that this disruption would be causing to them, the management personnel at Nokia sensed that this disruption would be of longer duration than promised. They (Nokia management) forced the Philips management to provide them with alternate sourcing of the chips by other manufacturing plants owned by Philips. Thus, the demands at the Nokia were met. Meanwhile, Ericsson were still waiting for the parts to arrive from Philips. This loss in time proved crucial to Ericsson and they incurred tremendous losses. Due to the proactive decision making and identifying the implications of the disruption by the Nokia personnel, they achieved profit and the trust of customers over their rivals Ericsson. 

\citep{Fujimoto2011} has outlined the importance of having a resilient supply chain by mentioning the case of the 1995 earthquake in Japan. The earthquake had massive impacts on the supply chains of Toyota and other companies in the affected areas. During the disaster, many of the facilities and plants that comprised of the parts provider for brakes and audio parts were completely levelled. The restoration of these plants were going to take a very long time which meant the production of the vehicles were going to come to a halt. Rather than just waiting for the plants to get restored, Toyota deployed personnel to the aid of getting the plants back to its functioning ability. In the meantime they had also developed a team that would search for alternate suppliers so that the production of the vehicles could be resumed as quickly as possible. This proved to extremely beneficial for Toyota as they reduced the down time to a few days rather than weeks. Thus, resiliency was incorporated by Toyota in their supply chain for having the minimum losses and getting the production of vehicles get back to the original level. Due to this, the importance of having a resilient supply chain came into light and various companies have started taking the initiatives to impart it in their supply chains.

\citep{Falasca2008} have done a considerable amount of research in the assessment of supply chain resilience. They developed a supply chain and made it undergo certain disruptions to study the effect of disruptions and to design the countermeasures to reduce the ramifications of the disruption. They have regarded supply chain structure as a number nodes that are inter-connected to each other. They have expressed the aspects of node density, node complexity, and node criticality as the three pillars that define the supply chain structure. If the disruption occurs in any of these pillars of the supply chain, then the entire supply chain would tremble. They have also mentioned that disruptions are more likely to take place in the supply chains having a significant amount of critical nodes. The method of simulation has been suggested as a methodology to replicate the supply chain structure and to understand its behavior under disruption. The cause-effect analysis of the measures taken to make the supply chain resilient can be obtained through the use of simulation. This research is used as the foundation for using simulation as a methodology in our study. This research also signifies resilience during the event of disruption in a supply chain.

\citep{Li2018} have developed an agent-based simulation model which focuses on fortifying the facility within a supply chain during disruptions. They have made use of the p-median problem model for choosing the facility which has the least distance for transportation. The RIMF method has been used to select the protective resources at the lowest costs in order to have the least impact during the disruption. In their study, they have created a supply chain consisting of 100 demand points in the United States of America. Out of these 100, 20 demand points are chosen at random to act as the distributors.  There are two scenarios for the occurrence of disruptions which have been developed in this study. In the first scenario, the three most important distributors in the network are fortified based on their importance score. In the second scenario, the four most important distributors are fortified. Once the disruptions occur at the distributors and that facility is destroyed, the p-median and RIMF methods help in choosing the facility such that the distance travelled is as minimum as possible, and the recovery of the destroyed facility at the lowest costs incurred. The most noteworthy points to be taken from this study is the use of agent-based simulation to study the effect of disruptions on the supply chain, the most cost effective method to resolve the disruptions, and the alternate facility selection having the least distance travelled.

The importance of the resilience and robustness can be easily observed from the research papers. Also, the role that simulation software plays an important role in quantifying the significance of these two terms in supply chain by giving the visual representation of the supply chain under disruptive conditions. Thus, we have made use of agent-based simulation in our study.

