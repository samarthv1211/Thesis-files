\chapter{Literature Review} \label{ch:litreview}

There has been research in the comprehension of the concepts of resiliency and robustness in the supply chain and its significance to a certain extent. Many firms have faced the adverse effects of disruption on the performance of their supply chains. The loss in terms of money, time, and the trust of the customers have forced the managers and owners to turn to the recourse of having resiliency and robustness incorporated in their initial design phase of the supply chain. The structure of the literature review body is in the following manner: a brief discussion about the ramifications of disruption, the significance of resilience in the supply chain, and the implementation of simulation in supply chain as a tool for evaluation of the resilience strategies.
\section{The Ramifications of Disruption}
The management of disruptions and the decisions that are made during the occurrence of a disruption plays a pivotal role in defining the performance and quality of the system. This can prove to have an advantage over the competitors. An example of this is the approach and decision-making skills of Nokia when the radio frequency chip manufacturing plant of Philips electronics was caught on fire (\citeauthor{latour2001trial}, \citeyear{latour2001trial}). The management personnel of Nokia addressed the ripple effect of the disruption and prevented the spreading of disruption. Due to the proactive decision making and identifying the implications of the disruption, the Nokia achieved profit and the trust of customers over their rivals Ericsson. 

In a network design for supply chain , it is of utmost importance to identify the sources of the disruptions  for enhancing the strategic operations (\citeauthor{klibi2010design}, \citeyear{klibi2010design}). They have stated that there is no modeling done on the comprehension of intensity of disruption in the supply chain network. There are two types of uncertainties that can affect the performance of the supply chain : Natural, and targeted attacks. The behavior and intensity of the disruption are directly associated with the density, complexity and nodes criticality of the supply chain (\citeauthor{craighead2007severity}, \citeyear{craighead2007severity}). The cost for facility rebuilding is taken care of through the insurance. The focus should not be on the repairing of the facility, but it should be towards the alternatives for keeping the supply chain to maintain a continuous operation. The damages incurred to the supply chain should be determined in terms of parameters such as capacity loss, supply loss or demand surge (\citeauthor{sheffi2005resilient}, \citeyear{sheffi2005resilient}).

\section{Previous work in Supply Chain Resilience}
The importance of having a resilient supply chain can be understood from the case of the 1995 earthquake in Japan (\citeauthor{Fujimoto2011}, \citeyear{Fujimoto2011}). The earthquake had massive impacts on the supply chains of Toyota and other companies in the affected areas. Toyota helped in providing personnel to the restoration process and simultaneously searched for alternate suppliers to keep the production continuous. The responsiveness is of utmost importance in enhancing the resilience when the customer vs. seller trade-offs are a crucial part of the supply chain (\citeauthor{saenz2018aligning}, \citeyear{saenz2018aligning}). In a supply chain, the resilience can be considered as a concoction of operational capabilities, integration capabilities, flexibility capabilities, and external capabilities (\citeauthor{brusset2017supply}, \citeyear{brusset2017supply}). Although these capabilities aid in making the supply chain resilient for a specific case study, it does not help in devising the generic solution for implementing resiliency. The resilience can be incorporated in the supply chain through integrating engineering recoverability into the system the same way engineering resilience is incorporated (\citeauthor{li2014engineering}, \citeyear{li2014engineering}). The recoverability and resilience of a system are directly related to each other. The fail safe mechanism (\citeauthor{Brandon-JonesE.;SquireB.;Autry2014}, \citeyear{Brandon-JonesE.;SquireB.;Autry2014}) makes the supply chain both adaptive and highly responsive to uncertainties. Thus, the supply chain becomes resilient as well as robust. The drawback to this approach is the costs incurred due to additional inventory being held up at every facility. The supply chain resilience can be linked with logistics capabilities for improving the performance of the supply chain (\citeauthor{ponomarov2009understanding}, \citeyear{ponomarov2009understanding}). If the supply chain has the right logistics capabilities, then it can make the supply chain capable of responding to unprecedented events. The resilience in a supply chain can be considered from a network perspective where the disruptions affect the nodes/arcs within a supply network (\citeauthor{Kim2015}, \citeyear{Kim2015}).  

\section{Supply Chain Simulation Literature}
 The simulation tool can be implemented to evaluate the behavior of supply chains under conditions of uncertainties (\citeauthor{carvalho2012supply}, \citeyear{carvalho2012supply}). The drawbacks of this paper is the consideration of redundancy as a strategy for resiliency. It does not mitigate the negative effects of the disruption while suffering monetary losses. There was a lack of flexibility considerations while designing the supply chain. Also, this approach is just suitable for the case study they have considered and is not applicable for generic supply chain design purposes. The importance of a flexible supply chain has been well expressed for making the system resilient (\citeauthor{christopher2011supply},\citeyear{christopher2011supply}). As the uncertainty increases in the future, the value for flexibility will be increasing. It is an essential component to be considered while designing the supply chain. They have not mentioned any model which consists of flexibility incorporated in the designing aspect of the supply chain. Implementing the tool of simulation to exhibit the benefits of the flexible supply chain over a normal supply chain would reinforce the reason to shift to a flexible supply chain. The aspects of node density, node complexity, and node criticality are the three pillars that define the supply chain structure (\citeauthor{Falasca2008}, \citeyear{Falasca2008}). If the disruption occurs in any of these pillars of the supply chain, then the entire supply chain would tremble. It has also been mentioned that disruptions are more likely to take place in the supply chains having a significant amount of critical nodes. The cause-effect analysis of the measures taken to make the supply chain resilient can be obtained through the use of simulation. The supply chain redesign (\citeauthor{carvalho2012supply}, \citeyear{carvalho2012supply}) has been evaluated in the face of disruption and the performance measures which are used for comparing the different designs are lead time and total cost. The simulation tool helped in experimenting with various supply chain designs.
 An agent-based simulation model has been developed which focuses on fortifying the facility within a supply chain during disruptions (\citeauthor{Li2018}, \citeyear{Li2018}). The p-median problem model has been used for choosing the facility which has the least distance for transportation. The RIMF method has been used to select the protective resources at the lowest costs in order to have the least impact during the disruption. The most noteworthy points to be taken from this study is the use of agent-based simulation to study the effect of disruptions on the supply chain, the most cost effective method to resolve the disruptions, and the alternate facility selection having the least distance travelled.

The importance of the resilience and robustness can be easily observed from the research papers. Also, the role that simulation software plays an important role in quantifying the significance of these two terms in supply chain by giving the visual representation of the supply chain under disruptive conditions. Thus, we have made use of agent-based simulation in our study.
