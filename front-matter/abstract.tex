\chapter*{Abstract}\label{ch:abstract}
 It is not possible to design a system in which there are no failures. To make the system more efficient, it is important to design the system in such a way that once the failure has occurred, the system can be restored to its full functionality in a short span of time. For a system to perform this task, we need to make the system "resilient". The basic idea of resilience is the ability to recover from an occurred failure in a system so that the system is performing at its best level. Resilience can be applied to all types of systems. This study aims at enhancing the resiliency of Supply Chains in the event of disruption at various levels. Supply Chains are one of the most important aspects for the growth and welfare of any business. Supply chain, just like any other system, is prone to disruption. A supply chain disruption is an unanticipated event that slows down the normal flow or even stops the normal flow of materials with hampering effects to the members within the supply chain. For enhancing the resiliency, various tools such as contingency planning, key indicators identification, and simulation have been implemented through the aggregate dimensions. The simulation tool is incredibly helpful for designing different scenarios in the highly complex supply chains. This study also sheds light on how the disruptions are transmitted through the various members of the supply chain and through different levels. The contingency planning plays an important role in the time of disruption by providing alternate sourcing ideology so that the supply chain does not come at a halt and costs the firm losses in terms of money and time.